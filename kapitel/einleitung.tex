\newpage
\section{Einleitung}
Dies ist ein indirektes Zitat. \footcite[Vgl.][S. 12]{Mustermann2022} Zitate mit Fußnoten werden automatisch aufsteigend im Literaturverzeichnis angeordnet.
\newline
\newline
Eine Abkürzung, welche im Literaturverzeichnis auftaucht sieht wie folgt aus:
\newline
\newline
Development and Operations (DevOps) \nomenclature{DevOps}{Development and Operations}
\newline
\newline
Es können außerdem Grafiken eingefügt werden:
\begin{figure}[H]
\centering
\includegraphics[width=5cm]{abbildungen/eufh.jpeg} %Mit \linewidth kann ein Bild über die ganze Breite der Seite eingefügt werden
\caption[EUFH Logo]{EUFH Logo\protect\footnotemark}
\end{figure}
\footnotetext{{In Anlehnung an \emph{Mustermann, M.} et al., 2022, S. 123.}}
\nocite{Mustermann2022}
Grafiken werden automatisch im Abbildungsverzeichnis aufgenommen und aufsteigend sortiert.
\newline
\newline
Es können außerdem Listen angefertigt werden. Diese sehen wie folgt aus:
\begin{itemize}
    \item Punkt 1
    \item Punkt 2
    \item Punkt 3
\end{itemize}
\subsection{Problemstellung}
\subsection{Zielsetzung}
\subsection{Aufbau der Arbeit}